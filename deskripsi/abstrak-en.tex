% Mengubah keterangan `Abstract` ke bahasa indonesia.
% Hapus bagian ini untuk mengembalikan ke format awal.
% \renewcommand\abstractname{Abstrak}

\begin{abstract}

  % Ubah paragraf berikut sesuai dengan abstrak dari penelitian.
  This study aims to develop an autonomous wheelchair control system that can follow human movement in real-time. This system integrates the object detection algorithm \emph{YOLOv11} and body motion tracking using MediaPipe Pose, utilizing the optimal viewing angle as the basis for visual data collection. The camera is placed on the user's glasses to obtain the optimal viewing angle in detecting and tracking user movement. This system is designed to operate wirelessly using the ESP32 module, providing flexibility and efficiency in controlling wheelchair movement. Testing was conducted in a controlled environment to ensure the accuracy and speed of user movement detection and tracking. The results showed that the system can follow user movement accurately and responsively, making a significant contribution to the development of smarter and more independent health mobility technology.


\end{abstract}

% Mengubah keterangan `Index terms` ke bahasa indonesia.
% Hapus bagian ini untuk mengembalikan ke format awal.
% \renewcommand\IEEEkeywordsname{Kata kunci}

\begin{IEEEkeywords}

  % Ubah kata-kata berikut sesuai dengan kata kunci dari penelitian.
  Autonomous Wheelchair, YOLOv11, MediaPipe Pose, Optimal Viewing Angle, Motion Detection, Wireless Control, Health Mobility.

\end{IEEEkeywords}
