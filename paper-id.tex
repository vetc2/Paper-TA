% Harus dimuat terlebih dahulu, digunakan agar file PDF memiliki format karakter yang benar.
% Untuk informasi lebih lanjut, lihat https://ctan.org/pkg/cmap.
\RequirePackage{cmap}

% Format dokumen sebagai paper konferensi menggunakan aturan IEEEtran terbaru (v1.8b).
% Untuk informasi lebih lanjut, lihat http://www.michaelshell.org/tex/ieeetran/.
\documentclass[a4paper, conference]{IEEEtran}

% Format encoding font dan input menjadi 8-bit UTF-8.
\usepackage[T1]{fontenc}
\usepackage[utf8]{inputenc}
\usepackage{amsmath}

% Digunakan untuk mengatur margin dokumen.
\usepackage{textcomp}

% Format bahasa menjadi bahasa indonesia dan inggris.
\usepackage[indonesian]{babel}

% Digunakan untuk tujuan demonstrasi.
\usepackage{mwe}

% Digunakan untuk menampilkan font dengan style yang lebih baik.
\usepackage[zerostyle=b,scaled=.75]{newtxtt}

% Digunakan untuk menampilkan tabel dengan style yang lebih baik.
\usepackage{booktabs}
\usepackage[table,xcdraw]{xcolor}
% Digunakan untuk menampilkan gambar pada dokumen.
\usepackage{graphicx}

% Digunakan untuk menampilkan potongan kode.
\usepackage{listings}
\lstset{
  basicstyle=\ttfamily,
  columns=fixed,
  basewidth=.5em,
  xleftmargin=0.5cm,
  captionpos=b
}

\usepackage{amsmath}
\usepackage{amssymb}
\usepackage{physics}
\usepackage{txfonts} % Font times
\usepackage{lipsum}
\usepackage{longtable}
\usepackage{tabularx}
\usepackage{wrapfig}
\usepackage{float}

\usepackage{tikz}
\usepackage{relsize}
\usetikzlibrary{shapes.geometric, arrows, positioning, decorations.pathreplacing, calc}

\tikzset{
  person/.style={
    minimum height=1.6cm, text centered, append after command={
      \pgfextra{
        \draw[thick, fill=green!30] (\tikzlastnode.north) circle (0.4cm);
        \draw[thick] (\tikzlastnode.north) ++(0,-0.4) -- ++(0,-0.8) -- ++(-0.4,-0.4); % Tubuh dan kaki kiri
        \draw[thick] (\tikzlastnode.north) ++(0,-0.4) -- ++(0,-0.8) -- ++(0.4,-0.4);  % Tubuh dan kaki kanan
        \draw[thick] (\tikzlastnode.north) ++(0,-0.4) -- ++(-0.4,-0.4); % Tangan kiri
        \draw[thick] (\tikzlastnode.north) ++(0,-0.4) -- ++(0.4,-0.4);  % Tangan kanan
      }
    }
  },
  motor/.style={
    draw, inner sep=0pt, append after command={
      \pgfextra{
        \draw[thick] (\tikzlastnode.south west) -- (\tikzlastnode.south east)
                    -- ++(0.5,1.5) -- ++(-3,0) -- cycle; % Menggambar trapesium manual
      }
    }
  },
  startstop/.style={ellipse, minimum width=3cm, minimum height=1cm, text centered, draw=black, fill=red!30},
  connector/.style={circle, minimum width=1cm, minimum height=1cm, text centered, draw=black, fill=green!30},
  io/.style={trapezium, trapezium left angle=70, trapezium right angle=110, minimum width=1.5cm, minimum height=1cm, 
                text centered, text width=2.5cm, draw=black, fill=blue!30},
  process/.style={rectangle, minimum width=3cm, minimum height=1cm, text centered, text width=4cm, draw=black, fill=orange!30},
  decision/.style={diamond, aspect=2, minimum width=3cm, minimum height=1cm, inner sep=0, text centered, text width=3cm, draw=black, fill=yellow!30},
  arrow/.style={thick,->,>=stealth, rounded corners},
  layer/.style={rectangle, rounded corners, minimum width=3cm, minimum height=1cm, text centered, align=center, draw=black, fill=blue!30},
  label/.style={align=right, font=\large, anchor=north},
  cbs/.style={rectangle, rounded corners, minimum width=3cm, minimum height=1cm, text centered, align=center, draw=black, fill=blue!20},
  c3k2/.style={rectangle, rounded corners, minimum width=3cm, minimum height=1cm, text centered, align=center, draw=black, fill=yellow!30},
  upsample/.style={rectangle, rounded corners, minimum width=3cm, minimum height=1cm, text centered, draw=black, fill=red!30},
  concat/.style={rectangle, rounded corners, draw=black, fill=blue!50, minimum width=2cm, text centered, minimum height=1cm},
  detect/.style={rectangle, rounded corners, minimum width=1.5cm, minimum height=1cm, text centered, draw=black},
  conv2D/.style={rectangle, rounded corners, minimum width=2cm, minimum height=1cm, text centered, draw=black, fill=magenta!50}
}
% Digunakan agar backticks (`) dapat dirender pada PDF.
% Untuk informasi lebih lanjut, lihat https://tex.stackexchange.com/a/341057/9075.
\usepackage{upquote}

% Digunakan untuk menyeimbangkan bagian akhir dokumen dengan dua kolom.
\usepackage{balance}

% Kapitalisasi caption tabel
\usepackage{caption}
\captionsetup[table]{
    justification=centering, % Memusatkan caption
    labelsep=newline, % Memisahkan label "TABLE 1" dengan judul dengan baris baru
    textfont={sc}, % Membuat teks menjadi kapital
    labelfont={sc} % Membuat teks menjadi kapital
}


% Digunakan untuk menampilkan pustaka.
\usepackage[square,comma,numbers,sort&compress]{natbib}

% Mengubah format ukuran teks pada natbib.
\renewcommand{\bibfont}{\normalfont\footnotesize}

% Jika melebihi 3 penulis dapat dilakukan linebreakend 
\makeatletter
\newcommand{\linebreakand}{%
  \end{@IEEEauthorhalign}
  \hfill\mbox{}\par
  \mbox{}\hfill\begin{@IEEEauthorhalign}
}
\makeatother

% Menambah nama penulis ketika menggunakan perintah \citet.
% Untuk informasi lebih lanjut, lihat https://tex.stackexchange.com/a/76075/9075.
\usepackage{etoolbox}
\makeatletter
\patchcmd{\NAT@test}{\else \NAT@nm}{\else \NAT@hyper@{\NAT@nm}}{}{}
\makeatother

% Digunakan untuk melakukan linewrap pada pustaka dengan url yang panjang
% jika terdapat hyphens
\usepackage[hyphens]{url}

% Digunakan untuk menambah hyperlink pada referensi.
\usepackage{hyperref}

% Menonaktifkan warna dan bookmark pada hyperref.
\hypersetup{hidelinks,
  colorlinks=true,
  allcolors=black,
  pdfstartview=Fit,
  breaklinks=true
}

% Digunakan untuk membenarkan hyperref pada gambar.
\usepackage[all]{hypcap}

% Digunakan untuk menampilkan beberapa gambar
%\usepackage[caption=false,font=footnotesize]{subfig}
\usepackage{multirow}
\usepackage{longtable}
\usepackage{graphicx}
\usepackage{subcaption}
\usepackage{stfloats}
\usepackage{float}
% nama
\newcommand{\name}{Aldi Fahmi Sihotang}
\newcommand{\authorname}{Sihotang, Aldi Fahmi}
\newcommand{\nickname}{Aldi}
\newcommand{\advisor}{Dr. Eko Mulyanto Yuniarno, S.T., M.T.}
\newcommand{\coadvisor}{Dr. Supeno Mardi Susiki Nugroho,S.T.,M.T.}

% identitas
\newcommand{\nrp}{0721 18 4000 0039}
\newcommand{\advisornip}{19680601 199512 1 009}
\newcommand{\coadvisornip}{19700313199512 1 001}
\newcommand{\email}{07211840000039@student.its.ac.id}
\newcommand{\advisoremail}{ekomulyanto@ee.its.ac.id}
\newcommand{\coadvisoremail}{mardi@its.ac.id}

% judul
\newcommand{\tatitle}{PENGEMBANGAN KURSI RODA OTONOM UNTUK MENGIKUTI MANUSIA BERBASIS \emph{YOLOv11}}
\newcommand{\engtatitle}{YOLOv11-BASED AUTONOMOUS WHEELCHAIR DEVELOPMENT FOR FOLLOWING HUMAN}
% tempat
\newcommand{\place}{Surabaya}   

% jurusan
\newcommand{\studyprogram}{Teknik Komputer}
\newcommand{\engstudyprogram}{Computer Engineering}

% fakultas
\newcommand{\faculty}{Teknologi Elektro dan Informatika Cerdas}
\newcommand{\engfaculty}{Intelligence Electrical and Informatics Technology}

% singkatan fakultas
\newcommand{\facultyshort}{FTEIC}
\newcommand{\engfacultyshort}{ELECTICS}

% departemen
\newcommand{\department}{Teknik Komputer}
\newcommand{\engdepartment}{Computer Engineering}

\input{pustaka/tanda-hubung.tex}

\begin{document}

\input{deskripsi/judul.tex}
% Mengubah keterangan `Abstract` ke bahasa indonesia.
% Hapus bagian ini untuk mengembalikan ke format awal.
\renewcommand\abstractname{Abstrak}

\begin{abstract}

  % Ubah paragraf berikut sesuai dengan abstrak dari penelitian.
  Penelitian ini bertujuan untuk mengembangkan sistem kendali kursi roda otonom yang dapat mengikuti pergerakan manusia secara real-time. Sistem ini mengintegrasikan algoritma deteksi objek \emph{YOLOv11} dan pelacakan gerakan tubuh menggunakan \emph{MediaPipe Pose}, dengan memanfaatkan sudut pandang optimal sebagai dasar pengambilan data visual. Kamera ditempatkan pada kacamata pengguna untuk mendapatkan sudut pandang yang optimal dalam mendeteksi dan melacak pergerakan pengguna. Sistem ini dirancang agar dapat beroperasi secara nirkabel menggunakan modul \emph{ESP32}, memberikan fleksibilitas dan efisiensi dalam mengendalikan pergerakan kursi roda. Pengujian dilakukan dalam lingkungan terkendali untuk memastikan keakuratan dan kecepatan deteksi serta pelacakan gerakan pengguna. Hasil penelitian menunjukkan bahwa sistem dapat mengikuti pergerakan pengguna dengan akurat dan responsif, memberikan kontribusi signifikan terhadap pengembangan teknologi mobilitas kesehatan yang lebih cerdas dan mandiri.

\end{abstract}

% Mengubah keterangan `Index terms` ke bahasa indonesia.
% Hapus bagian ini untuk mengembalikan ke format awal.
\renewcommand\IEEEkeywordsname{Kata kunci}

\begin{IEEEkeywords}

  % Ubah kata-kata berikut sesuai dengan kata kunci dari penelitian.
  Kursi Roda Otonom, YOLOv11, MediaPipe Pose, Sudut Pandang Optimal, Deteksi Gerakan, Kendali Nirkabel, Mobilitas Kesehatan

\end{IEEEkeywords}


% Ubah bagian berikut sesuai dengan konten-konten yang akan dimasukkan pada dokumen
% Ubah judul dan label berikut sesuai dengan yang diinginkan.
\section{Pendahuluan}
\label{sec:pendahuluan}

Teknologi \emph{Internet of Things (IoT)} dan \emph{deep learning} semakin banyak diadopsi dalam pengembangan sistem kendali otonom, khususnya untuk aplikasi kesehatan dan mobilitas. Kursi roda otonom merupakan salah satu solusi yang dapat meningkatkan kualitas hidup penyandang disabilitas dengan memberikan kemandirian dalam bergerak. Salah satu teknologi utama yang dapat mendukung pengembangan kursi roda otonom adalah modul \emph{ESP32}. Berdasarkan penelitian yang dilakukan oleh \emph{Ekatama} (2024), \emph{ESP32} mampu mengendalikan perangkat secara nirkabel dengan memanfaatkan konektivitas \emph{Wi-Fi} dan \emph{Bluetooth}, memberikan fleksibilitas yang lebih besar dalam mengontrol kursi roda. Penggunaan modul ini memungkinkan pengguna untuk mengontrol kursi roda secara efektif tanpa perlu mengandalkan kontrol manual yang terbatas \cite{ekatama2024perancangan}.

Selain aspek kendali nirkabel, sistem otonom yang mengikuti pergerakan pengguna membutuhkan teknologi yang mampu mendeteksi gerakan tubuh secara real-time. Penelitian oleh \emph{Wijaya et al.} (2022) telah menunjukkan keandalan algoritma \emph{YOLO V3} dalam mendeteksi objek secara cepat dan akurat, yang dapat digunakan untuk melacak pergerakan manusia. Namun, dalam penelitian ini, algoritma \emph{YOLOv11}, versi terbaru dari \emph{YOLO}, diintegrasikan dengan \emph{MediaPipe Pose}, sebuah framework yang dapat mendeteksi dan melacak posisi tubuh manusia. Kombinasi kedua teknologi ini memungkinkan sistem untuk secara akurat mengikuti gerakan pengguna berdasarkan sudut pandang optimal yang diperoleh dari kamera yang dipasang pada kacamata pengguna. Dengan pendekatan ini, kursi roda dapat mengikuti pergerakan pengguna secara alami, mendukung otonomi yang lebih tinggi \cite{wijaya2022deteksi}.

Sistem kursi roda otonom yang dikembangkan dalam penelitian ini memanfaatkan modul \emph{ESP32} sebagai perangkat keras utama yang mengontrol keseluruhan sistem secara nirkabel. Dalam penelitian yang dilakukan oleh \emph{Narwaria et al.} (2024), \emph{ESP32-CAM} telah terbukti mampu menangkap dan mengolah data visual dengan efisiensi tinggi, yang dapat diterapkan untuk berbagai aplikasi berbasis \emph{IoT}. Sistem yang dikembangkan tidak hanya dirancang untuk mendeteksi objek, tetapi lebih berfokus pada pelacakan gerakan tubuh pengguna melalui teknologi \emph{MediaPipe Pose}. Dengan integrasi ini, sistem dapat memastikan bahwa pergerakan kursi roda tetap sejalan dengan pergerakan pengguna tanpa memerlukan input manual tambahan \cite{10696374}.

Dengan latar belakang ini, proyek ini bertujuan untuk mengembangkan sistem kursi roda otonom yang dapat mengikuti pergerakan pengguna secara real-time, dengan rancangan dan implementasi sistem kendali kursi roda yang dapat mengikuti pergerakan pengguna dengan memanfaatkan sudut pandang optimal dari kamera yang ditempatkan pada kacamata.

\input{konten/id/2-tinjauan-pustaka.tex}
\input{konten/id/3-desain-dan-implementasi.tex}
% Ubah judul dan label berikut sesuai dengan yang diinginkan.
\section{Hasil dan Pembahasan}
\label{sec:hasildanpembahasan}

Pada penelitian ini dipaparkan skenario pengujian yang dilakukan, termasuk kondisi lingkungan pengujian, perangkat keras dan perangkat lunak yang digunakan, serta parameter-parameter yang diuji. Informasi detail mengenai konfigurasi dan prosedur pengujian disajikan agar hasil pengujian dapat direplikasi dengan konsisten.

\subsection{Hasil Pengujian Performa Menggunakan Confusion Matrix}
\label{subsec:hasilperformaconfisionMatrix}

Bagian ini membahas hasil pengujian performa deteksi objek menggunakan Confusion Matrix. Nilai seperti True Positive, True Negative, False Positive, dan False Negative dianalisis untuk menilai akurasi dan efektivitas deteksi.

\subsection{Pengujian Berdasarkan FPS}
\label{subsec:pengujianberdasarkanfps}

Pengujian ini dilakukan untuk menganalisis kecepatan pemrosesan sistem dalam satuan Frame per Second (FPS). FPS yang tinggi menunjukkan bahwa sistem dapat bekerja dengan baik secara real-time. Grafik hasil pengujian disertakan untuk memudahkan visualisasi performa.

\subsection{Pengujian Berdasarkan Response Time}
\label{subsec:pengujianberdasarkanresponsetime}

Bagian ini mengukur waktu yang dibutuhkan sistem untuk merespons setiap perubahan lingkungan atau perintah yang diterima. Response Time sangat penting untuk mengukur responsivitas sistem dalam skenario dinamis.

\subsection{Performa Keberhasilan Tracking}
\label{subsec:performakeberhasiltracking}

Bagian ini mengevaluasi keberhasilan sistem dalam melakukan tracking terhadap objek target. Tingkat keberhasilan dianalisis untuk mengetahui keandalan sistem dalam berbagai skenario.

\subsection{Pengujian Kesesuaian Jarak Deteksi}
\label{subsec:pengujiankesesuaianjarakdeteksi}

Pengujian dilakukan untuk mengukur kemampuan sistem dalam mendeteksi objek pada berbagai jarak  dan menganalisis performa sistem pada jarak-jarak tersebut.

\subsection{Performa Pergerakan Mengikuti Objek}
\label{subsec:performaakurasiobjek}

Analisis dilakukan untuk mengukur seberapa akurat sistem dapat mengikuti objek target, termasuk mempertahankan jarak yang tepat dan tidak kehilangan objek dalam berbagai kondisi.

\subsubsection{Saat berada didalam frame}
\label{subsubsec:dalamframe}

Bagian ini membahas kemampuan sistem dalam mengikuti objek saat berada didalam tangkapan layar. Tantangan yang dihadapi dan solusi yang diterapkan diuraikan.

\subsubsection{Saat berada diluar frame}
\label{subsubsec:luarframe}

Bagian ini membahas kemampuan sistem dalam mengikuti objek saat berada diluar frame. Tantangan yang dihadapi dan solusi yang diterapkan diuraikan.

\subsubsection{Belok kiri saat berada didalam frame}
\label{subsubsec:belokkiri}

Bagian ini membahas kemampuan sistem dalam mengikuti objek saat berbelok ke kiri, mirip dengan bagian sebelumnya.

\subsubsection{Belok Kanan saat berada didalam frame}
\label{subsubsec:belokkanan}

Bagian ini membahas kemampuan sistem dalam mengikuti objek saat berbelok ke kanan, mirip dengan bagian sebelumnya.

\subsection{Performa Keberhasilan Mengikuti}
\label{subsec:performamengikuti}

Bagian ini menguji apakah sistem dapat mengikuti objek pada lajur khusus, seperti jalur sempit atau berbelok yang membutuhkan manuver khusus.

\subsection{Pembahasan Hasil}
\label{subsec:pembahasanhasil}

Bagian ini membahas hasil-hasil pengujian secara keseluruhan dan memberikan insight mengenai performa sistem.

\subsubsection{Performa Deteksi Objek}
\label{subsec:performadeteksiobjek}

Bagian ini menjelaskan kekuatan dan kelemahan deteksi objek yang ditemukan selama pengujian.

\subsubsection{Kecepatan Pemrosesan (FPS)}
\label{subsec:kecepatanpemrosesan}

Menganalisis kecepatan pemrosesan secara keseluruhan dan pengaruhnya terhadap performa sistem real-time.

\subsubsection{Response Time}
\label{subsec:responsetime}

Membahas hasil pengujian response time dan faktor-faktor yang mempengaruhi waktu respons.

\subsubsection{Performa Keberhasilan Tracking}
\label{subsec:performatracking}

Membahas tingkat keberhasilan sistem dalam melacak objek dan kondisi yang mempengaruhi performa.

\subsubsection{Kesesuaian Jarak Deteksi}
\label{subsec:kesesuaianjarak}

Diskusi tentang performa sistem dalam mendeteksi objek pada berbagai jarak yang telah diuji.

\subsubsection{Performa Pergerakan Mengikuti Objek}
\label{subsec:akurasiikutiobjek}

Evaluasi akurasi dalam mengikuti objek target, termasuk kondisi yang dapat menyebabkan kegagalan.

\subsubsection{Performa Keberhasilan Mengikuti}
\label{subsec:keberhasilanmengikuti}

Menguraikan keberhasilan sistem dalam mengikuti objek saat menghadapi belokan dan lajur khusus.
\input{konten/id/5-kesimpulan.tex}
% Ucapan terima kasih jika ada
\input{konten/id/6-ucapan-terima-kasih.tex}

% Menampilkan daftar pustaka dengan format IEEE
\bibliographystyle{IEEEtran}
\bibliography{pustaka/pustaka.bib}

% Menyeimbangkan bagian akhir di kedua kolom
\balance

\end{document}