% Ubah judul dan label berikut sesuai dengan yang diinginkan.
\section{Kesimpulan}
\label{sec:kesimpulan}

% Ubah paragraf-paragraf pada bagian ini sesuai dengan yang diinginkan.

Berdasarkan hasil pengujian yang telah dilakukan, dapat diambil kesimpulan sebagai berikut:

\begin{enumerate}
  \item Model dengan Metrics tertinggi yang telah di-training dengan berbagai konfigurasi adalah model dengan skor mAP di IoU 0.5 tertinggi sebesar 81.85\%. Adapun nilai yang didapatkan ini sudah cukup baik untuk melakukan penghindaran dilihat pada hasil performa penghindaran yang sangat baik.
  \item Performa NUC dalam pengujian FPS menghasilkan Nilai yang lebih rendah ketimbang Laptop pribadi Penulis dengan selisih 7.029 fps.
  \item Rata - rata delay yang didapatkan pada pengujian adalah sekitar 0.2494 detik dan rata- rata nilai inference yang didapatkan 139.4899 ms atau 0.1394 detik.
  \item Hasil menunjukkan bahwa deteksi menggunakan \emph{Bounding box} dan Landmark bahu lebih akurat pada jarak yang lebih jauh (150 cm dan 100cm), sedangkan landmark lengan lebih akurat pada jarak yang lebih dekat (50 cm). Dimana rata-rata \emph{difference} terbaik bounding box yaitu pada jarak 150cm sebesar 3.2 cm, rata-rata \emph{difference} landmark bahu terbaik yaitu pada jarak 100cm sebesar 2.2 cm dan rata-rata \emph{difference} landmark lengan terbaik yaitu pada jarak 50cm sebesar 1.93 cm.
  \item Hasil Performa Deteksi menunjukan hasil yang memuaskan dalam 30 sampel pengujian. Dengan presentasi keberhasilan sebesar 100\% yang menunjukan bahwa sistem yang dibuat dapat menghindari manusia dengan sangat baik.

\end{enumerate}

\section{Saran}
\label{chap:saran}

Untuk pengembangan lebih lanjut pada penelitian selanjutnya, adapun saran yang bisa diberikan antara lain:

\begin{enumerate}

  \item Variasi dataset yang lebih ditingkat untuk meningkatkan performa deteksi
  \item Menggunakan Device yang memiliki performa yang lebih baik untuk fps yang lebih tinggi
  \item Meningkatkan performa grid deteksi dengan melakukan penyesuaian yang lebih detail untuk pemetaan yang lebih baik lagi.
  \item Menggunakan pendingin device saat melakukan pengujian di ruangan terbuka agar tidak mengalami penurunan performa.
\end{enumerate}