% Ubah judul dan label berikut sesuai dengan yang diinginkan.
\section{Pendahuluan}
\label{sec:pendahuluan}

Teknologi \emph{Internet of Things (IoT)} dan \emph{deep learning} semakin banyak diadopsi dalam pengembangan sistem kendali otonom, khususnya untuk aplikasi kesehatan dan mobilitas. Kursi roda otonom merupakan salah satu solusi yang dapat meningkatkan kualitas hidup penyandang disabilitas dengan memberikan kemandirian dalam bergerak. Salah satu teknologi utama yang dapat mendukung pengembangan kursi roda otonom adalah modul \emph{ESP32}. Berdasarkan penelitian yang dilakukan oleh \emph{Ekatama} (2024), \emph{ESP32} mampu mengendalikan perangkat secara nirkabel dengan memanfaatkan konektivitas \emph{Wi-Fi} dan \emph{Bluetooth}, memberikan fleksibilitas yang lebih besar dalam mengontrol kursi roda. Penggunaan modul ini memungkinkan pengguna untuk mengontrol kursi roda secara efektif tanpa perlu mengandalkan kontrol manual yang terbatas \cite{ekatama2024perancangan}.

Selain aspek kendali nirkabel, sistem otonom yang mengikuti pergerakan pengguna membutuhkan teknologi yang mampu mendeteksi gerakan tubuh secara real-time. Penelitian oleh \emph{Wijaya et al.} (2022) telah menunjukkan keandalan algoritma \emph{YOLO V3} dalam mendeteksi objek secara cepat dan akurat, yang dapat digunakan untuk melacak pergerakan manusia. Namun, dalam penelitian ini, algoritma \emph{YOLOv11}, versi terbaru dari \emph{YOLO}, diintegrasikan dengan \emph{MediaPipe Pose}, sebuah framework yang dapat mendeteksi dan melacak posisi tubuh manusia. Kombinasi kedua teknologi ini memungkinkan sistem untuk secara akurat mengikuti gerakan pengguna berdasarkan sudut pandang optimal yang diperoleh dari kamera yang dipasang pada kacamata pengguna. Dengan pendekatan ini, kursi roda dapat mengikuti pergerakan pengguna secara alami, mendukung otonomi yang lebih tinggi \cite{wijaya2022deteksi}.

Sistem kursi roda otonom yang dikembangkan dalam penelitian ini memanfaatkan modul \emph{ESP32} sebagai perangkat keras utama yang mengontrol keseluruhan sistem secara nirkabel. Dalam penelitian yang dilakukan oleh \emph{Narwaria et al.} (2024), \emph{ESP32-CAM} telah terbukti mampu menangkap dan mengolah data visual dengan efisiensi tinggi, yang dapat diterapkan untuk berbagai aplikasi berbasis \emph{IoT}. Sistem yang dikembangkan tidak hanya dirancang untuk mendeteksi objek, tetapi lebih berfokus pada pelacakan gerakan tubuh pengguna melalui teknologi \emph{MediaPipe Pose}. Dengan integrasi ini, sistem dapat memastikan bahwa pergerakan kursi roda tetap sejalan dengan pergerakan pengguna tanpa memerlukan input manual tambahan \cite{10696374}.

Dengan latar belakang ini, proyek ini bertujuan untuk mengembangkan sistem kursi roda otonom yang dapat mengikuti pergerakan pengguna secara real-time, dengan rancangan dan implementasi sistem kendali kursi roda yang dapat mengikuti pergerakan pengguna dengan memanfaatkan sudut pandang optimal dari kamera yang ditempatkan pada kacamata.
