% Ubah judul dan label berikut sesuai dengan yang diinginkan.
\section{Hasil dan Pembahasan}
\label{sec:hasildanpembahasan}

Pada penelitian ini dipaparkan skenario pengujian yang dilakukan, termasuk kondisi lingkungan pengujian, perangkat keras dan perangkat lunak yang digunakan, serta parameter-parameter yang diuji. Informasi detail mengenai konfigurasi dan prosedur pengujian disajikan agar hasil pengujian dapat direplikasi dengan konsisten.

\subsection{Hasil Pengujian Performa Menggunakan Confusion Matrix}
\label{subsec:hasilperformaconfisionMatrix}

Bagian ini membahas hasil pengujian performa deteksi objek menggunakan Confusion Matrix. Nilai seperti True Positive, True Negative, False Positive, dan False Negative dianalisis untuk menilai akurasi dan efektivitas deteksi.

\subsection{Pengujian Berdasarkan FPS}
\label{subsec:pengujianberdasarkanfps}

Pengujian ini dilakukan untuk menganalisis kecepatan pemrosesan sistem dalam satuan Frame per Second (FPS). FPS yang tinggi menunjukkan bahwa sistem dapat bekerja dengan baik secara real-time. Grafik hasil pengujian disertakan untuk memudahkan visualisasi performa.

\subsection{Pengujian Berdasarkan Response Time}
\label{subsec:pengujianberdasarkanresponsetime}

Bagian ini mengukur waktu yang dibutuhkan sistem untuk merespons setiap perubahan lingkungan atau perintah yang diterima. Response Time sangat penting untuk mengukur responsivitas sistem dalam skenario dinamis.

\subsection{Performa Keberhasilan Tracking}
\label{subsec:performakeberhasiltracking}

Bagian ini mengevaluasi keberhasilan sistem dalam melakukan tracking terhadap objek target. Tingkat keberhasilan dianalisis untuk mengetahui keandalan sistem dalam berbagai skenario.

\subsection{Pengujian Kesesuaian Jarak Deteksi}
\label{subsec:pengujiankesesuaianjarakdeteksi}

Pengujian dilakukan untuk mengukur kemampuan sistem dalam mendeteksi objek pada berbagai jarak  dan menganalisis performa sistem pada jarak-jarak tersebut.

\subsection{Performa Pergerakan Mengikuti Objek}
\label{subsec:performaakurasiobjek}

Analisis dilakukan untuk mengukur seberapa akurat sistem dapat mengikuti objek target, termasuk mempertahankan jarak yang tepat dan tidak kehilangan objek dalam berbagai kondisi.

\subsubsection{Saat berada didalam frame}
\label{subsubsec:dalamframe}

Bagian ini membahas kemampuan sistem dalam mengikuti objek saat berada didalam tangkapan layar. Tantangan yang dihadapi dan solusi yang diterapkan diuraikan.

\subsubsection{Saat berada diluar frame}
\label{subsubsec:luarframe}

Bagian ini membahas kemampuan sistem dalam mengikuti objek saat berada diluar frame. Tantangan yang dihadapi dan solusi yang diterapkan diuraikan.

\subsubsection{Belok kiri saat berada didalam frame}
\label{subsubsec:belokkiri}

Bagian ini membahas kemampuan sistem dalam mengikuti objek saat berbelok ke kiri, mirip dengan bagian sebelumnya.

\subsubsection{Belok Kanan saat berada didalam frame}
\label{subsubsec:belokkanan}

Bagian ini membahas kemampuan sistem dalam mengikuti objek saat berbelok ke kanan, mirip dengan bagian sebelumnya.

\subsection{Performa Keberhasilan Mengikuti}
\label{subsec:performamengikuti}

Bagian ini menguji apakah sistem dapat mengikuti objek pada lajur khusus, seperti jalur sempit atau berbelok yang membutuhkan manuver khusus.

\subsection{Pembahasan Hasil}
\label{subsec:pembahasanhasil}

Bagian ini membahas hasil-hasil pengujian secara keseluruhan dan memberikan insight mengenai performa sistem.

\subsubsection{Performa Deteksi Objek}
\label{subsec:performadeteksiobjek}

Bagian ini menjelaskan kekuatan dan kelemahan deteksi objek yang ditemukan selama pengujian.

\subsubsection{Kecepatan Pemrosesan (FPS)}
\label{subsec:kecepatanpemrosesan}

Menganalisis kecepatan pemrosesan secara keseluruhan dan pengaruhnya terhadap performa sistem real-time.

\subsubsection{Response Time}
\label{subsec:responsetime}

Membahas hasil pengujian response time dan faktor-faktor yang mempengaruhi waktu respons.

\subsubsection{Performa Keberhasilan Tracking}
\label{subsec:performatracking}

Membahas tingkat keberhasilan sistem dalam melacak objek dan kondisi yang mempengaruhi performa.

\subsubsection{Kesesuaian Jarak Deteksi}
\label{subsec:kesesuaianjarak}

Diskusi tentang performa sistem dalam mendeteksi objek pada berbagai jarak yang telah diuji.

\subsubsection{Performa Pergerakan Mengikuti Objek}
\label{subsec:akurasiikutiobjek}

Evaluasi akurasi dalam mengikuti objek target, termasuk kondisi yang dapat menyebabkan kegagalan.

\subsubsection{Performa Keberhasilan Mengikuti}
\label{subsec:keberhasilanmengikuti}

Menguraikan keberhasilan sistem dalam mengikuti objek saat menghadapi belokan dan lajur khusus.