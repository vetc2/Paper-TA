% Ubah judul dan label berikut sesuai dengan yang diinginkan.
\section{Introduction}
\label{sec:introduction}

% Ubah paragraf-paragraf pada bagian ini sesuai dengan yang diinginkan.

The integration of \emph{Internet of Things (IoT)} and \emph{deep learning} technologies is increasingly adopted in the development of autonomous control systems, particularly for healthcare and mobility applications. An autonomous wheelchair is one solution that can improve the quality of life for people with disabilities by providing independence in movement. One of the key technologies that can support the development of an autonomous wheelchair is the \emph{ESP32} module. According to research conducted by \emph{Ekatama} (2024), the \emph{ESP32} can control devices wirelessly using \emph{Wi-Fi} and \emph{Bluetooth} connectivity, offering greater flexibility in controlling the wheelchair. The use of this module allows users to effectively control the wheelchair without relying on limited manual control \cite{ekatama2024perancangan}.

In addition to wireless control aspects, an autonomous system that follows the user's movements requires technology capable of detecting body movements in real-time. Research by \emph{Wijaya et al.} (2022) has demonstrated the reliability of the \emph{YOLO V3} algorithm in detecting objects quickly and accurately, which can be used to track human movements. However, in this study, the \emph{YOLOv11} algorithm, the latest version of \emph{YOLO}, is integrated with \emph{MediaPipe Pose}, a framework that can detect and track human body positions. The combination of these two technologies allows the system to accurately follow the user's movements based on the optimal viewpoint obtained from a camera mounted on the user's glasses. With this approach, the wheelchair can naturally follow the user's movements, supporting higher autonomy \cite{wijaya2022deteksi}.

The autonomous wheelchair system developed in this research utilizes the \emph{ESP32} module as the main hardware that controls the entire system wirelessly. In research conducted by \emph{Narwaria et al.} (2024), the \emph{ESP32-CAM} has been proven to capture and process visual data with high efficiency, which can be applied to various \emph{IoT}-based applications. The developed system is not only designed to detect objects but focuses more on tracking the user's body movements through \emph{MediaPipe Pose} technology. With this integration, the system can ensure that the wheelchair's movements remain aligned with the user's movements without requiring additional manual input \cite{10696374}.

With this background, this project aims to develop an autonomous wheelchair system that can follow the user's movements in real-time, with the design and implementation of a wheelchair control system that can follow the user's movements by utilizing the optimal viewpoint from a camera mounted on the glasses.
